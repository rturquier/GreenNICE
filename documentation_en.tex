% Options for packages loaded elsewhere
\PassOptionsToPackage{unicode}{hyperref}
\PassOptionsToPackage{hyphens}{url}
%
\documentclass[
]{article}
\usepackage{geometry}
 \geometry{
 a4paper,
 total={150mm,240mm},
 left=30mm,
 top=20mm,
 }
\usepackage{amsmath,amssymb}
\usepackage{longtable}
\usepackage{array}
\usepackage{booktabs} 
\usepackage{secnum}
\usepackage{iftex}
\usepackage{hyperref}
\usepackage{fancyvrb}
\usepackage{mdframed}
\ifPDFTeX
  \usepackage[T1]{fontenc}
  \usepackage[utf8]{inputenc}
  \usepackage{textcomp} % provide euro and other symbols
\else % if luatex or xetex
  \usepackage{unicode-math} % this also loads fontspec
  \defaultfontfeatures{Scale=MatchLowercase}
  \defaultfontfeatures[\rmfamily]{Ligatures=TeX,Scale=1}
\fi
\usepackage{lmodern}
\ifPDFTeX\else
  % xetex/luatex font selection
\fi
% Use upquote if available, for straight quotes in verbatim environments
\IfFileExists{upquote.sty}{\usepackage{upquote}}{}
\IfFileExists{microtype.sty}{% use microtype if available
  \usepackage[]{microtype}
  \UseMicrotypeSet[protrusion]{basicmath} % disable protrusion for tt fonts
}{}
\makeatletter
\@ifundefined{KOMAClassName}{% if non-KOMA class
  \IfFileExists{parskip.sty}{%
    \usepackage{parskip}
  }{% else
    \setlength{\parindent}{0pt}
    \setlength{\parskip}{6pt plus 2pt minus 1pt}}
}{% if KOMA class
  \KOMAoptions{parskip=half}}
\makeatother
\usepackage{xcolor}
\usepackage{color}
\usepackage{fancyvrb}
\newcommand{\VerbBar}{|}
\newcommand{\VERB}{\Verb[commandchars=\\\{\}]}
\DefineVerbatimEnvironment{Highlighting}{Verbatim}{commandchars=\\\{\}}
% Add ',fontsize=\small' for more characters per line
\newenvironment{Shaded}{}{}
\newcommand{\AlertTok}[1]{\textcolor[rgb]{1.00,0.00,0.00}{\textbf{#1}}}
\newcommand{\AnnotationTok}[1]{\textcolor[rgb]{0.38,0.63,0.69}{\textbf{\textit{#1}}}}
\newcommand{\AttributeTok}[1]{\textcolor[rgb]{0.49,0.56,0.16}{#1}}
\newcommand{\BaseNTok}[1]{\textcolor[rgb]{0.25,0.63,0.44}{#1}}
\newcommand{\BuiltInTok}[1]{\textcolor[rgb]{0.00,0.50,0.00}{#1}}
\newcommand{\CharTok}[1]{\textcolor[rgb]{0.25,0.44,0.63}{#1}}
\newcommand{\CommentTok}[1]{\textcolor[rgb]{0.38,0.63,0.69}{\textit{#1}}}
\newcommand{\CommentVarTok}[1]{\textcolor[rgb]{0.38,0.63,0.69}{\textbf{\textit{#1}}}}
\newcommand{\ConstantTok}[1]{\textcolor[rgb]{0.53,0.00,0.00}{#1}}
\newcommand{\ControlFlowTok}[1]{\textcolor[rgb]{0.00,0.44,0.13}{\textbf{#1}}}
\newcommand{\DataTypeTok}[1]{\textcolor[rgb]{0.56,0.13,0.00}{#1}}
\newcommand{\DecValTok}[1]{\textcolor[rgb]{0.25,0.63,0.44}{#1}}
\newcommand{\DocumentationTok}[1]{\textcolor[rgb]{0.73,0.13,0.13}{\textit{#1}}}
\newcommand{\ErrorTok}[1]{\textcolor[rgb]{1.00,0.00,0.00}{\textbf{#1}}}
\newcommand{\ExtensionTok}[1]{#1}
\newcommand{\FloatTok}[1]{\textcolor[rgb]{0.25,0.63,0.44}{#1}}
\newcommand{\FunctionTok}[1]{\textcolor[rgb]{0.02,0.16,0.49}{#1}}
\newcommand{\ImportTok}[1]{\textcolor[rgb]{0.00,0.50,0.00}{\textbf{#1}}}
\newcommand{\InformationTok}[1]{\textcolor[rgb]{0.38,0.63,0.69}{\textbf{\textit{#1}}}}
\newcommand{\KeywordTok}[1]{\textcolor[rgb]{0.00,0.44,0.13}{\textbf{#1}}}
\newcommand{\NormalTok}[1]{#1}
\newcommand{\OperatorTok}[1]{\textcolor[rgb]{0.40,0.40,0.40}{#1}}
\newcommand{\OtherTok}[1]{\textcolor[rgb]{0.00,0.44,0.13}{#1}}
\newcommand{\PreprocessorTok}[1]{\textcolor[rgb]{0.74,0.48,0.00}{#1}}
\newcommand{\RegionMarkerTok}[1]{#1}
\newcommand{\SpecialCharTok}[1]{\textcolor[rgb]{0.25,0.44,0.63}{#1}}
\newcommand{\SpecialStringTok}[1]{\textcolor[rgb]{0.73,0.40,0.53}{#1}}
\newcommand{\StringTok}[1]{\textcolor[rgb]{0.25,0.44,0.63}{#1}}
\newcommand{\VariableTok}[1]{\textcolor[rgb]{0.10,0.09,0.49}{#1}}
\newcommand{\VerbatimStringTok}[1]{\textcolor[rgb]{0.25,0.44,0.63}{#1}}
\newcommand{\WarningTok}[1]{\textcolor[rgb]{0.38,0.63,0.69}{\textbf{\textit{#1}}}}
\setlength{\emergencystretch}{3em} % prevent overfull lines
\providecommand{\tightlist}{%
  \setlength{\itemsep}{0pt}\setlength{\parskip}{0pt}}
\setcounter{secnumdepth}{-\maxdimen} % remove section numbering
\usepackage{amsmath}
\usepackage{amsfonts}
\usepackage{amssymb}
\usepackage{bookmark}
\IfFileExists{xurl.sty}{\usepackage{xurl}}{} % add URL line breaks if available
\urlstyle{same}
%\hypersetup{
%  hidelinks,
%  pdfcreator={LaTeX via pandoc}}

\author{T. O. Da Costa \\{\small  Update by M Young-Brun 30/08/2024} }
\date{}
\title{NICE 2020 model --- code documentation}

\setcounter{secnumdepth}{3}

\begin{document}

\maketitle
\tableofcontents

\newpage

\section{Model - Indexes}

\renewcommand{\arraystretch}{1.5}
\begin{longtable}{|c|c|c|}
  \hline
  \textbf{Symbol Name} & \textbf{Model Name} & \textbf{Short description} \\
  \hline
  \endhead
  $c$ & \texttt{country} & Countries included in the model \\

  $q$ & \texttt{quantile} & Index of income quantiles \\
  
  
  $rwpp$ &\texttt{regionwpp} & Regions defined from the World Population Prospects database (UN) \\
  
  $t$ & \texttt{time} & Time index \\

  \hline
\end{longtable}



\section{Model - Parameters}

\textit{N.B.} : \$ = 2017 US dollars. pers. = person. yr = year.

\renewcommand{\arraystretch}{1.5}
\begin{longtable}{|p{1.5in}|p{2.5in}|p{0.9in}|p{0.7in}|}
  \hline
  \textbf{Name} & \textbf{Short description} & \textbf{Unit} & \textbf{Value} \\
  \hline
  \endhead
  \(\beta1\) & Linear damage coefficient on temperature. This parameter represents the direct impact of the temperature anomaly on the economy & $T^{-\beta2}$ & 0.0236 \\
  \(\beta1\_{KW}[t,c]\) & Linear damage coefficient on local temperature anomaly for Kalkuhl and Wenz based damage function & $T^{-1}$ & — \\
  \(\beta2\) & Power damage coefficient on temperature. This parameter captures the non-linear effects of temperature anomalies on the economy & — & 2 \\
  \(\beta2\_{KW}[t,c]\) & Quadratic damage coefficient on local temperature anomaly for Kalkuhl and Wenz based damage function & $T^{-2}$ & — \\
  \(\beta\_{temp}[c]\) & Temperature scaling coefficients, which translate global temperature anomalies into country-level temperature anomalies & — & — \\
  \(\eta\) & Inequality aversion. & — & 1.5 (by default) \\
  \(\sigma[t,c]\) & Emissions output ratio. This parameter is used for modelling emissions intensity as a function of economic activity & GtCO$_2$/($10^6$ \$) & — \\
  \(\theta2\) & Exponent of abatement cost function (DICE-2023 value) & — & 2.6 \\
  \(\mu\_input[t,c]\) & Input mitigation rate, used with option 3 ``country\_abatement\_rate'' & \% & — \\
  \(\text{control\_regime}\) & Switch for emissions control regime ; 1 = ``global\_carbon\_tax'', 2 = ``country\_carbon\_tax'', 3 = ``country\_abatement\_rate'' & — & 3 (by default) \\

  \(\text{damage\_elasticity}\) & Income elasticity of climate damages (1 = proportional to income) & — & 0.85 \\
  \(\text{depk[t,c]}\) & Depreciation rate on capital & \% & — \\
  \(\text{elasticity}\) & Income elasticity with respect to climate damage, mitigation costs, etc. It can either be \texttt{damage\_elasticity} or \texttt{CO2\_income\_elasticity} & — & — \\
  \(\text{elasticity\_intercept[t]}\) & Intercept term for estimating income elasticity & — & 3.22\footnotemark[1] \\
  \(\text{elasticity\_slope[t]}\) & Slope term for estimating income elasticity & — & -0.2\footnotemark[1]\\
  \(\text{global\_carbon\_tax[t]}\) & Global carbon tax & \$/tCO$_2$ & — \\
  \(\text{global\_recycle\_share[c]}\) & Share of country revenues that are recycled globally in the form of international transfers (1 = 100\%) & \% & 1 (by default) \\
  \(\text{global\_temperature[t]}\) = \(\text{temp\_anomaly[t]}\) & Global average surface temperature excess (above pre-industrial {[}year 1750{]} level) & °C & — \\
  \(\text{income\_shares[c,q]}\) & An array of income shares by quantile (where rows represent countries and columns represent quantiles) & — & — \\
  \(\text{increase\_value}\) & The annual tax increase value. By default, it is equal to \texttt{tax\_start\_value}, which means that the tax increases by its initial value each year & \$/tCO$_2$ & — \\
  \(\text{k0[c]}\) & Initial level of capital. Determines the starting point for capital accumulation & $10^6$ \$/yr & — \\
  \(\text{l[t,c]}\) & Labor/population. This parameter represents either the population or the available workforce for production of a given country & $10^3$ pers. & — \\

  \(\text{local\_temp\_anomaly[t,c]}\) & Country-level average surface temperature anomaly (above pre-industrial {[}year 1750{]} level) & °C & — \\
  \(\text{lost\_revenue\_share}\) & Portion of carbon tax revenue that is lost and cannot be recycled (1 = 100\% of revenue lost, 0 = no revenue lost) & \% & 0 (by default) \\

  \(\text{mapcrwpp[c]}\) & Mapping from country index to UN WPP region index & — &183 countries to 20 regions \\
  \(\text{max\_study\_gdp}\) & Maximum value of GDP per capita observed in elasticity studies & \$/pers. & 48892 \\
  \(\text{min\_study\_gdp}\) & Minimum value of GDP per capita observed in elasticity studies & \$/pers. & 647 \\
  \(\text{nb\_country}\) & Number of countries & — & 183 \\
  \(\text{nb\_quantile}\) & Number of quantiles & — & 10 (by default) \\
  \(\text{pbacktime[t]}\) & \emph{Backstop price} from DICE 2023 & \$/tCO$_2$ & — \\
  \(\text{quantile\_income}\) \(\text{\_shares[t,c,q]}\) & Income shares of quantile & — & — \\
  \(\text{recycle\_share[c,q]}\) & Share of carbon tax revenues recycled to each quantile & — & $\frac{1}{\mbox{nb\_quantile}}$ (by default)\\
  \(\text{reference\_carbon\_tax[t]}\) & Reference carbon tax & \$/tCO$_2$ & — \\
  \(\text{reference\_country\_index}\) & Reference country index & — & USA \\
  \(\text{s[t,c]}\) & Savings rate & \% & — \\
  \(\text{share}\) & Capital's share of production. This parameter is global and affects the distribution of income between capital and labor & \% & 0.3 \\

  \(\text{switch\_global}\) \(\text{\_pc\_recycle}\) & Boolean, carbon tax revenues are recycled globally on an equal per capita basis & — & 0 or 1 \\
  \(\text{switch\_recycle}\) & Boolean for recycling carbon tax revenue & — & 0 or 1 \\
  \(\text{switch\_scope\_recycle}\) & Boolean, carbon tax revenues are recycled at national (0) or global (1) level & — & 0 or 1 \\
  \(\text{tax\_start\_value}\) & The initial value of the carbon tax & \$/tCO$_2$ & Depends on the chosen carbon tax pathway \\
  \(\text{temp\_anomaly[t]}\) = \(\text{global\_temperature[t]}\) & Global average surface temperature excess (above pre-industrial {[}year 1750{]} level) & °C & — \\
  \(\text{tfp[t,c]}\) & Total factor productivity & — & — \\
  \(\text{year\_model\_end}\) & The end of the model. If it is less than \texttt{year\_tax\_end}, the last tax value is repeated up to this year & yr & 2300 (by default) \\
  \(\text{year\_step}\) & The step in years between two tax values & yr & 1 (by default) \\
  \(\text{year\_tax\_end}\) & The last year for which to calculate the tax & yr & 2200 (by default) \\
  \(\text{year\_tax\_start}\) & The first year of the tax increase & yr & 2020 (by default) \\
  \hline
\end{longtable}

\footnotetext[1]{Results from the  meta-regression based on study results to calculate elasticity vs. ln gdp per capita relationship.}


\section{Model - Variables}

\begin{longtable}{|p{1.5in}|p{3in}|p{0.9in}|p{0.5in}|}
  \hline
  \textbf{Name} & \textbf{Short description} & \textbf{Unit} \\
  \hline
  \endhead
  \(\mu[t,c]\) & GHG emissions mitigation rate & \% \\
 
  \(\theta1[t,c]\) & Multiplicative parameter of abatement cost function. Equal to ABATEFRAC at 100\% mitigation & — \\
  \text{ABATECOST[t,c]} & Cost of emission reductions & $10^6$ \$/yr \\
  \text{ABATEFRAC[t,c]} & Cost of emission reductions as a share of gross economic output & \% \\
  \text{abatement\_cost} \text{\_dist[t,c,q]} & Share of the distribution of abatement costs per quantile & — \\
  \text{C[t,c]} & Country consumption & $10^6$ \$/yr \\
  \text{carbon\_tax\_dist[t,c,q]} & Shares of the distribution of CO2 tax burden per quantile & — \\
  \text{cons\_EDE\_country[t,c]} & Consumption equivalent to equitably distributed well-being in a given country & $10^3$ \$/pers/yr \\
  \text{cons\_EDE\_global[t]} & Global consumption equivalent to equitably distributed well-being & $10^3$ \$/pers/yr \\
  \text{cons\_EDE\_rwpp[t,rwpp]} & Consumption equivalent to equitably distributed well-being for WPP regions & $10^3$ \$/pers/yr \\
  \text{country\_carbon\_tax[t,c]} & CO2 tax rate & \$/tCO$_2$ \\
  \text{country\_pc\_dividend[t,c]} & Total fiscal dividends per person, including all international monetary transfers & $10^3$ \$/pers/yr \\
  \text{country\_pc\_dividend} \text{\_domestic\_transfers[t,c]} & Fiscal dividends per person from domestic redistribution, i.e., within a country & $10^3$ \$/pers/yr \\
  \text{country\_pc\_dividend} \text{\_global\_transfers[t,c]} & Tax dividends per person from international transfers & $10^3$ \$/pers/yr \\
  \text{CPC[t,c]} & Country level consumption per capita & $10^3$ \$/pers/yr \\
  \text{CPC\_post[t,c]} & Country level consumption per capita after recycling & $10^3$ \$/pers/yr \\
  \text{CPC\_post\_global[t]} & World consumption per capita after recycling & $10^3$ \$/pers/yr  \\
  \text{CPC\_post\_rwpp[t,rwpp]} & Regional per capita consumption after recycling & $10^3$ \$/pers/yr \\
  \text{CPC\_rwp[t,rwpp]} & Regional level consumption per capita & $10^3$ \$/pers/yr \\
  \text{DAMFRAC[t,c]} & Country-level damages as a share of net GDP based on global temperatures & \% \\
  \text{damage\_dist[t,c,q]} & Share of the distribution of climate damage per quantile & — \\
  \text{E\_gtco2[t,c]} & Country-level total GHG emissions & $10^9$ tCO2/yr \\
  \text{E\_Global\_gtco2[t]} & Global emissions (sum of all country emissions) & $10^9$ tCO2/yr \\
  \text{E\_Global\_gtc[t]} & Global emissions in units of gigatonnes of carbon, giving compatible units with FAIR & $10^9$ tC/yr \\
  \text{GLOBAL\_ABATEFRAC} \text{\_full\_abatement[t]} & Global \texttt{ABATEFRAC[t]} in case of full mitigation & \% \\
  \text{global\_gini\_cons[t]} & Gini index of world consumption & — \\
  \text{global\_revenue[t]} & Carbon tax revenue, derived from the total recycled revenue of all countries & \$/yr \\
  \text{global\_pc\_revenue[t]} & Carbon tax revenue per person, derived from the total recycled revenue of all countries & $10^3$ \$/pers/yr \\
  \text{gini\_cons[t,c]} & Gini index of country consumption & — \\
  \text{gini\_cons\_rwpp[t,rwpp]} & Gini index of regional consumption & — \\
  \text{I[t,c]} & Investment & $10^6$ \$/yr \\
  \text{K[t,c]} & Capital & $10^6$ \$/yr \\
  \text{l\_rwpp[t,rwpp]} & Regional population & $10^3$ pers. \\
  \text{LOCAL\_} \text{DAMFRAC\_KW[t,c]} & Country-level damages as a share of net GDP based on local temperatures and on Kalkuhl \& Wenz & \% \\
  \text{local\_temperature[t,c]} & Excess temperature at country level (above pre-industrial {[}year 1750{]} level)& °C \\

  \text{qcpc\_base[t,c,q]} & Consumption per quantile per capita before damage, before abatement cost, before tax & $10^3$ \$pers/yr \\
  \text{qcpc\_post} \text{\_damage\_abatement[t,c,q]} & Consumption per quantile per capita after damage, after abatement & $10^3$ \$/pers/yr \\
  \text{qc\_post\_recycle[t,c,q]} & Consumption per quantile per capita after recycling the carbon tax to each quantile & $10^3$ \$/pers/yr \\
  \text{qcpc\_post\_tax[t,c,q]} & Consumption per quantile per capita after subtraction of the carbon tax & $10^3$ \$/pers/yr \\
  \text{qcpc\_share[t,c,q]} & Proportion of consumption per quantile per capita & \% \\
  \text{revenue\_recycled} \text{\_global\_level[t]} & Recycle a share \texttt{global\_recycle\_share} of \texttt{tax\_revenue} at global level & $10^3$ \$/yr \\
  \text{tax\_pc\_revenue[t,c]} & Carbon tax revenue per capita from country emissions & $10^3$ \$/pers/yr \\
  \text{tax\_revenue[t,c]} & Carbon tax revenue for a given country & \$/yr \\
  \text{tax\_values[t]} & Array containing the carbon tax values over time, until the last year of tax defined & — \\
  \text{sum\_qcpc\_} \text{post\_recycle[t,c]} & Sum of quantiles post consumption per capita  & $10^3$ \$ / (pers per quant) / yr\\
  \text{total\_tax\_pc\_revenue[t]} & Total carbon tax revenue per person, sum of tax revenues in all countries per person & $10^3$ \$/pers/yr \\
  \text{total\_tax\_revenue[t]} & Total carbon tax revenue, sum of tax revenues in all countries & $10^3$ \$/yr \\
  
  \text{updated\_quantile} \text{\_distribution[t,c,q]} & An array which contains the updated quantile share distribution for each country, considering the given income elasticity & — \\
  \text{welfare\_country[t,c]} & Welfare for countries & — \\
  \text{welfare\_global[t]} & Global welfare & — \\
  \text{welfare\_rwpp[t,rwpp]} & Welfare in a given WPP region & — \\
  \text{YGROSS[t,c]} & Gross output & $10^6$ \$/yr \\
  \text{YGROSS\_global[t]} & Global gross output, represents the sum of all countries' gross production & $10^{12}$ \$/yr \\
  \text{Y[t,c]} & Output net of damages and abatement costs & $10^6$ \$/yr \\
  \text{Y\_pc[t,c]} & Net GDP per capita after damages and mitigation costs & \$/pers/yr  \\
  \text{Y\_pc\_rwpp[t,rwpp]} & Regional per capita output net of abatement and damages & \$/pers/yr \\
  \hline
\end{longtable}

\section{In theory}

The goal is to maximize the following Social Welfare function : 
\begin{equation} 
W(\text{cons}(c,q,t)) = \sum_{cqt} \frac{l(c,q,t)}{(1+\rho)^t}\frac{\text{cons}(c,q,t)^{1-\eta}}{1 - \eta}
\end{equation}

\begin{itemize}
\item $W$ : social welfare
\item cons :  per capita consumption
\item $l$ : population
\item $\eta$ : inequality aversion (marginal utility elasticity)
\item $\rho$ : rate of pure time preference
\item $c$ : country
\item $q$ : quantile
\item $t$ : time
\end{itemize}


\section{abatement.jl}\label{abatement.jl}

This file defines the components needed to calculate the cost of
mitigation.

\subsection{Indexes}\label{indexes}

\begin{itemize}
\tightlist
\item
  \texttt{country}: Countries included in the model.

\end{itemize}

\subsection{Parameters}\label{parameters}

\subsubsection{Time-dependent}\label{time-dependent}

\begin{itemize}
\tightlist
\item
  \texttt{pbacktime}: \emph{backstop price} from DICE 2023, in USD2017
  per tCO2.
\item
  \texttt{global\_carbon\_tax}: Global carbon tax (USD2017 per tCO2).
\item
  \texttt{reference\_carbon\_tax}: Reference carbon tax (USD2017 per tCO2).
\end{itemize}

\subsubsection{Time and country
dependent}\label{time-and-country-dependent}

\begin{itemize}
\tightlist
\item
  \(\sigma\) : Emissions output ratio, GtCO2 per million of 2017 US dollars .
\item
  \texttt{YGROSS}: Gross output, measured in millions of 2017 US dollars per year.
\item
  \texttt{s}: Savings rate (\%).
\item
  \texttt{l}: Labor/population (thousands).
\item
  \(\mu\_\text{input}\) : Input mitigation rate, used with option 3
  ``country\_abatement\_rate''.
\end{itemize}

\subsubsection{Independent}\label{independent}

\begin{itemize}
\tightlist
\item
  \(\eta\) : Inequality aversion.
\item
  \(\theta2\) : Exponent of abatement cost function (DICE-2023 value).
\item
  \texttt{reference\_country\_index}: Reference country index.
\item
  \texttt{control\_regime}: Switch for emissions control regime ; 1 =
  ``global\_carbon\_tax'', 2 = ``country\_carbon\_tax'', 3 =
  ``country\_abatement\_rate''.
\end{itemize}

\subsection{Variables}\label{variables}

\subsubsection{Time and country
dependent}\label{time-and-country-dependent-1}

\begin{itemize}
\tightlist
\item
  \(\theta1\) : Multiplicative parameter of abatement cost function.
  Equal to ABATEFRAC at 100\% mitigation
\item
  \texttt{country\_carbon\_tax}: CO2 tax rate (2017 USD per tCO2)
\item
  \(\mu\) : GHG emissions mitigation rate (\(\mu \in [0;1]\)).
\item
  \texttt{ABATEFRAC}: Cost of emission reductions in share of gross
  output.
\item
  \texttt{ABATECOST}: Cost of emission reductions, in millions USD2017
  per year.
\end{itemize}

\subsubsection{Time-dependent}\label{time-dependent-1}

\begin{itemize}
\tightlist
\item
  \texttt{GLOBAL\_ABATEFRAC\_full\_abatement}: global \texttt{ABATEFRAC}
  in case of full mitigation (\emph{i.e.} \(\mu = 1\)).
\end{itemize}

\subsection{Equations}\label{equations}

\subsubsection{\texorpdfstring{Calculating $\theta1$ for each
country}{Calculating theta1 for each country}}\label{calculating-theta1-for-each-country}

For each country \texttt{c} in the \texttt{country} set, \(\theta1\) is
calculated as follows :

\begin{equation}
\theta1[t,c] = pbacktime[t] \times \frac{\sigma[t,c] \times 10^3}{\theta2}
\end{equation}


The \(10^3\) factor is used for converting \(\sigma\) into tCO2/USD2017.

Then, depending on the emission control regime specified by
\texttt{p.control\_regime}, different methods are used to calculate the
abatement rate \(\mu\) and the country-specific carbon tax
\texttt{country\_carbon\_tax}.

\paragraph{\texorpdfstring{a. Global carbon tax,
\texttt{control\_regime\ ==\ 1}}{a. Global carbon tax, control\_regime == 1}}\label{a.-global-carbon-tax-control_regime-1}

A global carbon tax is applied to all countries :

\begin{equation}
country\_carbon\_tax[t,c] = global\_carbon\_tax[t]
\end{equation}


\begin{equation}
 \mu[t,c] = min \left(max(0, (\frac{country\_carbon\_tax[t,c] \times \sigma[t,c] \times 10^3}{\theta1[t,c] \times \theta2})^{\frac{1}{\theta2 - 1}}), 1 \right) 
\end{equation}


\paragraph{\texorpdfstring{b. Carbon tax by country,
\texttt{control\_regime\ ==\ 2}}{b. Carbon tax by country, control\_regime == 2}}\label{b.-carbon-tax-by-country-control_regime-2}

Each country's carbon tax is calculated from the reference carbon tax,
adjusted according to the country's economic and demographic parameters.

\begin{multline}
country\_carbon\_tax[t,c] = reference\_carbon\_tax[t]\\ \times \left(\frac{1 - s[t,reference\_country\_index]}{1 - s[t,c]} \right)^{1-\eta}\\
\times \left(\frac{YGROSS[t,c]  }{YGROSS[t,reference\_country\_index] } \frac{l[t,reference\_country\_index]}{l[t,c]} \right)^\eta
\end{multline}


\begin{equation}
 \mu[t,c] = min \left(max(0, (\frac{country\_carbon\_tax[t,c] \times \sigma[t,c] \times 10^3}{\theta1[t,c] \times \theta2})^{\frac{1}{\theta2 - 1}}), 1 \right) 
\end{equation}

\paragraph{\texorpdfstring{c. Abatement rate by country,
\texttt{control\_regime\ ==\ 3}}{c. Abatement rate by country, control\_regime == 3}}\label{c.-abatement-rate-by-country-control_regime-3}

The abatement rate \(\mu\) is supplied directly by the user
(\(\mu[t,c] = \mu\_input[t,c]\)) and the country's carbon tax is
calculated according to this rate.

\begin{equation}
 country\_carbon\_tax[t,c] =  \frac{\theta1[t,c] \times \theta2}{\sigma[t,c] \times 10^3} * \mu^{\theta2 - 1}[t,c]
\end{equation}


\subsubsection{Calculation of abatement cost, abatement rate and overall
abatement rate in the case of full
abatement}\label{calculation-of-abatement-cost-abatement-rate-and-overall-abatement-rate-in-the-case-of-full-abatement}

For each country, the abatement rate \texttt{ABATEFRAC} and the
abatement cost \texttt{ABATECOST} are calculated as follows:

\begin{equation}
ABATEFRAC[t,c] = \theta1[t,c] \times \mu^{\theta2}[t,c] 
\end{equation}


\begin{equation}
ABATECOST[t,c] = YGROSS[t,c] \times ABATEFRAC[t,c]
\end{equation}


\begin{equation}
GLOBAL\_ABATEFRAC\_full\_abatement[t] = \frac{\sum_{c} (\theta1[t,c] \times YGROSS[t,c])}{\sum_{c} YGROSS[t,c]} 
\end{equation}


\section{damages.jl}\label{damages.jl}

This file calculates the economic damage linked to temperature
anomalies, as a share of GDP.

\subsection{Indexes}\label{indexes-1}

\begin{itemize}
\tightlist
\item
  \texttt{country}: Countries included in the model.
\item
  \texttt{regionwpp}: Regions WPP.
\end{itemize}

\subsection{Parameters}\label{parameters-1}


\subsubsection{Time and country
dependent}\label{time-and-country-dependent-2}

\begin{itemize}
\tightlist
\item
  \texttt{local\_temp\_anomaly}: Country-level average surface
  temperature anomaly (°C above pre-industrial {[}year 1750{]}).
\item
  \(\beta1\_{KW}\): Linear damage coefficient on local temperature
  anomaly for Kalkuhl and Wenz based damage function.
\item
  \(\beta2\_{KW}\): Quadratic damage coefficient on local temperature
  anomaly for Kalkuhl and Wenz based damage function.
\end{itemize}

\subsection{Variables}\label{variables-1}

\subsubsection{Time and country
dependent}\label{time-and-country-dependent-3}

\begin{itemize}
\tightlist
\item
  \texttt{LOCAL\_DAMFRAC\_KW}: Country-level damages as a share of net
  GDP based on local temperatures and on Kalkuhl \& Wenz.

\end{itemize}

\subsection{Equations}\label{equations-1}


\begin{multline}
 temp\_LOCAL\_DAMFRAC\_KW\_GROSS[t,c] = \beta1\_KW[c] \times local\_temp\_anomaly[t,c] \\
 + \beta2\_KW[c] \times local\_temp\_anomaly^2[t,c]
\end{multline}

\begin{multline}
 LOCAL\_DAMFRAC\_KW[t,c] = temp\_LOCAL\_DAMFRAC\_KW\_GROSS[t,c] / \\(1-temp\_LOCAL\_DAMFRAC\_KW\_GROSS[t,c])
\end{multline}

\section{emissions.jl}\label{emissions.jl}

This file allows to calculate greenhouse gas (GHG) emissions at global
and national level.

\subsection{Indexes}\label{indexes-2}

\begin{itemize}
\tightlist
\item
  \texttt{country}: Countries included in the model.
\item
  \texttt{regionwpp}: WPP regions.
\end{itemize}

\subsection{Parameters}\label{parameters-2}

\subsubsection{Country dependent}\label{country-dependent}

\begin{itemize}
\tightlist
\item
  \texttt{mapcrwpp}: Mapping from country index to WPP region index.
  This parameter is indexed by country, linking country data to specific WPP regions.
\end{itemize}

\subsubsection{Time and country
dependent}\label{time-and-country-dependent-4}

\begin{itemize}
\item
  \(\sigma\) : Emissions output ratio, GtCO2 per million USD2017. This
  parameter is used for modelling emissions intensity as a function of
  economic activity.
\item
  \texttt{YGROSS}: Gross output, in million USD2017 per year. This
  parameter represents the economic activity of each country.
\item
  \(\mu\) : GHG emissions control rate.
\end{itemize}

\subsection{Variables}\label{variables-2}

\subsubsection{Time-dependent}\label{time-dependent-3}

\begin{itemize}
\item
  \texttt{E\_Global\_gtco2}: Global emissions (sum of all country
  emissions), in GtCO2 per year.
\item
  \texttt{E\_Global\_gtc}: Global emissions in units of gigatonnes of
  carbon (GtC per year), giving compatible units with FAIR.
\end{itemize}

\subsubsection{Time and country
dependent}\label{time-and-country-dependent-5}

\begin{itemize}
\tightlist
\item
  \texttt{E\_gtco2}: Country level total greenhouse gas emissions, in
  GtCO2 per year.
\end{itemize}

\subsubsection{Time and WPP region
dependent}\label{time-and-WPP-region-dependent}

\begin{itemize}
\tightlist
\item
  \texttt{E\_gtco2\_rwpp}: Regional GHG emissions for the WPP regions, measured in GtCO2 per year.
\end{itemize}

\subsection{Equations}\label{equations-2}

\subsubsection{Emissions by country}\label{emissions-by-country}

For each country \texttt{c} in the \texttt{country} set, CO2 emissions
are calculated using the following equation:

\begin{equation}
 E_{\text{gtco2}}[t,c] = YGROSS[t,c] \cdot \sigma[t,c] \cdot (1-\mu[t,c]) 
\end{equation}


\subsubsection{Global emissions}\label{global-emissions}

Global CO2 emissions in year \texttt{t} are calculated as the sum of all
countries' emissions:

\begin{equation}
 E_{\text{Global\_gtco2}}[t] = \sum_{c} E_{\text{gtco2}}[t,c] 
\end{equation}


\subsubsection{Global emissions in Carbon Units
(GtC)}\label{global-emissions-in-carbon-units-gtc}

Global emissions are then converted into gigatons of carbon (GtC) using
the molecular conversion factor :

\begin{equation}
 E_{\text{Global\_gtc}}[t] = E_{\text{Global\_gtco2}}[t] \cdot \frac{12.01}{44.01} 
\end{equation}


This conversion is necessary for integration with the FAIR model, which
uses emissions in GtC.

\subsubsection{Regional emissions for WPP
regions}\label{regional-emissions-for-wpp-regions}

For each \texttt{rwpp} WPP region in the set of \texttt{regionwpp}
regions, regional CO2 emissions are calculated as the sum of the
emissions of the countries belonging to that region:

\begin{equation}
 E_{\text{gtco2\_rwpp}}[t,rwpp] = \sum_{c \in \text{rwpp}} E_{\text{gtco2}}[t,c] 
\end{equation}


This process makes it possible to aggregate CO2 emissions at different
levels of granularity.

\section{grosseconomy.jl}\label{grosseconomy.jl}

\subsection{Index}\label{index}

\begin{itemize}
\tightlist
\item
  \texttt{country}: Countries in the model.
\end{itemize}

\subsection{Parameters}\label{parameters-3}

\subsubsection{Country dependent}\label{country-dependent-1}

\begin{itemize}
\tightlist
\item
  \texttt{k0}: Initial level of capital. Determines the starting point
  for capital accumulation.
\end{itemize}

\subsubsection{Time and country
dependent}\label{time-and-country-dependent-6}

\begin{itemize}
\item
  \texttt{I}: Investment, in millions of
  2017 US dollars per year (1e6 USD2017/year).
\item
  \texttt{l}: Labor/population (workforce available for production).
\item
  \texttt{tfp}: Total factor productivity.
\item
  \texttt{depk}: Depreciation rate on capital. Represents the loss of
  capital value due to wear and tear and obsolescence.
\end{itemize}

\subsubsection{Independent}\label{independent-2}

\begin{itemize}
\tightlist
\item
  \texttt{share}: Capital's share of production. This parameter is
  global and affects the distribution of income between capital and
  labor.
\end{itemize}

\subsection{Variables}\label{variables-3}

\subsubsection{Time-dependent}\label{time-dependent-4}

\begin{itemize}
\tightlist
\item
  \texttt{YGROSS\_global}: Global gross output, measured in trillions of
  2017 US dollars per year (1e12 USD2017/year). Represents the sum of all countries'
  gross production.
\end{itemize}

\subsubsection{Time and country
dependent}\label{time-and-country-dependent-7}

\begin{itemize}
\item
  \texttt{YGROSS}: Gross output in each country, measured in millions of
  2017 US dollars per year.
\item
  \texttt{K}: Capital, in millions of USD2017 per year.
\end{itemize}

\subsection{Equations}\label{equations-3}

\subsubsection{\texorpdfstring{Capital
(\texttt{K})}{Capital (K)}}\label{capital-k}

The capital in each country is calculated for each period \texttt{t} and
for each country \texttt{c}.

\begin{itemize}
\tightlist
\item
  For the first period (\texttt{t} is the first period), capital is
  initialized to the value \texttt{k0} for country \texttt{c} :
\end{itemize}

\begin{equation}
 K[t,c] = k0[c]
\end{equation}


\begin{itemize}
\tightlist
\item
  For subsequent periods, capital is calculated by taking into account
  the previous year's capital depreciation and the previous year's
  investment:
\end{itemize}

\begin{equation}
 K[t,c] = (1 - depk[t,c]) \cdot K[t-1,c] + I[t-1,c] 
\end{equation}


\subsubsection{\texorpdfstring{Calculation of Gross Production by Country
(\texttt{YGROSS})}{Calculation of Gross Production by Country (YGROSS)}}\label{calculation-of-gross-production-by-country-ygross}

Gross output by country is calculated for each period \texttt{t} and for
each country \texttt{c} using the Cobb-Douglas production function
equation:

\begin{equation}
 YGROSS[t,c] = tfp[t,c] \cdot K[t,c]^{share} \cdot l[t,c]^{(1-share)} 
\end{equation}


\subsubsection{\texorpdfstring{Calculation of Global Gross Production
(\texttt{YGROSS\_global})}{Calculation of Global Gross Production (YGROSS\_global)}}\label{calculation-of-global-gross-production-ygross_global}

Global gross output is calculated for each period \texttt{t} by
aggregating the gross output of all countries and converting the result
into trillions of USD2017 :

\begin{equation}
 YGROSS\_global[t] = \frac{\sum_{c} YGROSS[t,c]}{10^6}
\end{equation}


\section{neteconomy.jl}\label{neteconomy.jl}

This file calculates the net savings by taking into account the
mitigation costs and damages associated with climate change.

\subsection{Indexes}\label{indexes-3}

\begin{itemize}
\tightlist
\item
  \texttt{country}: Countries included in the model.
\item
  \texttt{regionwpp}: Regions defined from the World Population Prospects database (UN).
\end{itemize}

\subsection{Parameters}\label{parameters-4}

\subsubsection{Country dependent}\label{country-dependent-2}

\begin{itemize}
\tightlist
\item
  \texttt{mapcrwpp}: Mapping from country index to WPP region index.
  This parameter is indexed by country, linking country data to specific
  regions defined by the UN WPP.
\end{itemize}

\subsubsection{Time and country
dependent}\label{time-and-country-dependent-8}

\begin{itemize}
\item
  \texttt{YGROSS}: Gross output in each country, measured in millions of
  2017 US dollars per year.
\item
  \texttt{ABATEFRAC}: Cost of emission reductions in share of gross
  output.
\item
  \texttt{LOCAL\_DAMFRAC\_KW}: Country-level damages as a share of net
  GDP based on local temperatures and on Kalkuhl \& Wenz.
\item
  \texttt{s}: Savings rate (\%).
\item
  \texttt{l}: Labor/population (in thousands).
\end{itemize}

\subsection{Variables}\label{variables-4}

\subsubsection{Time and country
dependent}\label{time-and-country-dependent-9}

\begin{itemize}
\item
  \texttt{Y}: Output net of damages and abatement costs, in millions
  USD2017 per year.
\item
  \texttt{C}: Country consumption, in millions USD2017 per year.
\item
  \texttt{CPC}: Country level consumption per capita, in thousands USD2017 per person per year.
\item
  \texttt{Y\_pc}: Per capita output net of abatement and damages, in
  USD2017 per person per year.
\item
  \texttt{I}: Investment, in millions of
   USD2017 per year.
\end{itemize}

\subsubsection{Time and WPP region dependent}\label{time-and-region-dependent}

\begin{itemize}
\tightlist
\item
  \texttt{C\_rwpp}: Regional consumption, in millions USD2017 per year.
\item
  \texttt{l\_rwpp}: Regional population (in thousands).
\item
  \texttt{CPC\_rwpp}: Regional level consumption per capita, in thousand
  USD2017 per person per year.
\item
  \texttt{Y\_pc\_rwpp}: Regional per capita output net of abatement and
  damages, in USD2017 per person per year.
\end{itemize}

\subsection{Equations}\label{equations-4}

\begin{itemize}
\tightlist
\item
  Regional consumption \texttt{C\_rwpp} is initialized to zero for the
  current time step.
\end{itemize}

\subsubsection{Country loop}\label{country-loop}

For each country \texttt{c} in the \texttt{country} set:

\begin{enumerate}
\def\labelenumi{\arabic{enumi}.}
\item
  \textbf{Net Output} calculation: Net output \texttt{Y} is calculated
  by adjusting gross output \texttt{YGROSS} for mitigation costs
  \texttt{ABATEFRAC} and damages \texttt{LOCAL\_DAMFRAC\_KW} according
  to equation :

  \begin{equation}
 Y[t,c] = \frac{(1 - ABATEFRAC[t,c])}{(1 + LOCAL\_DAMFRAC\_KW[t,c])} \cdot YGROSS[t,c] 
\end{equation}

\item
  \textbf{Investment} : Investment \texttt{I} is determined as a
  fraction \texttt{s} of net output \texttt{Y} :

  \begin{equation}
 I[t,c] = s[t,c] \cdot Y[t,c] 
\end{equation}

\item
  \textbf{Consumption by country} : Consumption \texttt{C} is calculated
  as net output \texttt{Y} minus investment \texttt{I}, except for the
  last period where it is equal to consumption in the previous period.

  \begin{equation}
 C[t,c] =
  \begin{cases}
  Y[t,c] - I[t,c] & \text{if } t \text{ is not the last period} \\
  C[t-1, c] & \text{otherwise}
  \end{cases}
  \end{equation}

\item
  \textbf{Consumption per capita} : Per capita consumption \texttt{CPC}
  is calculated by dividing consumption \texttt{C} by population
  \texttt{l}.

  \begin{equation}
 CPC[t,c] = \frac{C[t,c]}{l[t,c]}
\end{equation}

\item
  \textbf{Net output per head} : Net output per capita \texttt{Y\_pc} is
  calculated by adjusting net output \texttt{Y} for population
  \texttt{l}, with scaling to convert units.

  \begin{equation}
 Y\_pc[t,c] = \frac{Y[t,c]}{l[t,c]} \times 10^3
\end{equation}

\end{enumerate}

\subsubsection{Loop over WPP regions}\label{loop-over-wpp-regions}

For each region \texttt{rwpp} in the set of regions \texttt{regionwpp} :

\begin{enumerate}
\def\labelenumi{\arabic{enumi}.}
\item
  \textbf{Country indices} : The indices of the \texttt{c} countries
  belonging to the \texttt{rwpp} region are identified.
\item
  \textbf{Regional Consumption}:
  \begin{equation}
 C\_rwpp[t, rwpp] = \sum_{c \in rwpp} C[t, c]
\end{equation}

\item
  \textbf{Regional population} :
  \begin{equation}
 l\_rwpp[t, rwpp] = \sum_{c \in rwpp} l[t, c]
\end{equation}

\item
  \textbf{Regional per capita consumption} :
  \begin{equation}
 CPC\_rwpp[t,rwpp] = \frac{C[t,rwpp]}{l[t,rwpp]}
\end{equation}

\item
  \textbf{Regional net output per capita} :
  \begin{equation}
 Y\_pc\_rwpp[t,c] = \frac{\sum_{c \in rwpp}Y[t,c]}{\sum_{c \in rwpp}l[t,c]} \times 10^3 
\end{equation}

\end{enumerate}


\section{pattern\_scale.jl}\label{pattern_scale.jl}

File designed to calculate local temperature changes from global mean
temperature variations, using country-specific scaling coefficients.

\subsection{Index}\label{index-1}

\begin{itemize}
\tightlist
\item
  \texttt{country\ =\ Index()}: defines a \texttt{country} index for the
  modelled regions. This index is used for iterating over the different
  countries in the model.
\end{itemize}

\subsection{Parameters}\label{parameters-8}

\begin{itemize}
\item
  \(\beta\_{temp}\)\texttt{=\ Parameter(index={[}country{]})}: declares
  a \(\beta\_{temp}\) parameter which represents temperature scaling
  coefficients. These coefficients translate global temperature
  anomalies into country-level temperature anomalies. They are indexed
  by country.
\item
  \texttt{global\_temperature\ =\ Parameter(index={[}time{]})}: declares
  a \texttt{global\_temperature} parameter which represents the
  time-dependent global average temperature excess (in °C).
\end{itemize}

\subsection{Variables}\label{variables-5}

\begin{itemize}
\tightlist
\item
  \texttt{local\_temperature\ =\ Variable(index={[}time,\ country{]})}:
  declares a \texttt{local\_temperature} variable which represents the
  excess temperature at country level (in °C) according to country and
  year (time-dependent).
\end{itemize}

\subsection{Function}\label{function}

\begin{itemize}
\tightlist
\item
  \texttt{function\ run\_timestep(p,\ v,\ d,\ t)}: Defines the
  \texttt{run\_timestep} function which is called at each time step of
  the model. It takes as parameters the parameters (\texttt{p}), the
  variables (\texttt{v}), the dimensions (\texttt{d}), and the current
  time step (\texttt{t}).
\end{itemize}

\subsection{Equation}\label{equation}

For all countries, the local temperature is estimated as the product of
the scaling coefficient \(\beta\_{temp}\) and the global mean
temperature anomaly at time \texttt{t}.
\begin{equation}
 \text{local\_temperature}[t,c] = \beta\_temp[c] \cdot \text{global\_temperature}[t]
\end{equation}


\section{quantile\_recycle.jl}\label{quantile_recycle.jl}

This file calculates many variables (see \emph{infra}), including per
capita consumption, the poverty rate or the Gini index.

\subsection{Indices}\label{indices}

\begin{itemize}
\tightlist
\item
  \texttt{country}: Countries included in the model.
\item
  \texttt{regionwpp}: Regions defined from the World Population Prospects database (UN).
\item
  \texttt{quantile}: Index of income quantiles
\end{itemize}

\subsection{Parameters}\label{parameters-5}

\subsubsection{Time and country
dependent}\label{time-and-country-dependent-10}

\begin{itemize}
\tightlist
\item
  \texttt{ABATEFRAC}: Cost of CO2 emission reductions as share of gross
  economic output.
\item
  \texttt{LOCAL\_DAMFRAC\_KW}: National damages as a share of net GDP
  calculated from national temperatures.
\item
  \texttt{CPC}: Country level consumption per capita, in thousands of USD2017 per person per year.
\item
  \texttt{l}: Population of the country (in thousands).
\item
  \texttt{Y} : Net production after damage and mitigation costs, in
  millions USD2017 per year.
  \item
  \texttt{Y\_pc} : Net production per capita after damage and mitigation costs, in
  USD2017 pers person per year.
\item
  \texttt{country\_pc\_dividend}: Total per capita revenue from the
  carbon tax, including any international transfers, in thousands of
  USD2017 per person per year.
\item
  \texttt{tax\_pc\_revenue}: Carbon tax revenue per capita from country
  emissions, in thousands of USD2017 per person per year.
\end{itemize}

\subsubsection{Other}\label{other}

\begin{itemize}
\tightlist
\item
  \texttt{switch\_recycle}: Recycling of carbon tax revenues.
\item
  \texttt{nb\_quantile}: Number of quantiles (10 by default).
\item
  \texttt{min\_study\_gdp}, \texttt{max\_study\_gdp}: Minimum and
  maximum values of GDP per capita observed in elasticity studies (in
  dollars per person).
\item
  \texttt{elasticity\_intercept}, \texttt{elasticity\_slope}: Intercept
  and slope terms for estimating income elasticity (time-dependent).
\item
  \texttt{damage\_elasticity}: Elasticity of income in relation to
  climate damage (1 = proportional to income).

\item
  \texttt{quantile\_income\_shares}: Income shares of deciles, depending
  on time, country and quantiles.
\item
  \texttt{recycle\_share}: Share of carbon tax revenues recycled to each
  quantile, depending on country and quantiles.
\item
  \texttt{mapcrwpp}: Mapping from country index to WPP region index.
\end{itemize}

\subsection{Variables}\label{variables-6}

\subsubsection{Time-dependent}\label{time-dependent-5}

\begin{itemize}
\tightlist
\item
  \texttt{CPC\_post\_global}: World consumption per capita after
  recycling, in thousands of USD2017 per person per year.
\item
  \texttt{global\_gini\_cons}: Gini index of world consumption.

\end{itemize}

\subsubsection{Time and country
dependent}\label{time-and-country-dependent-11}

\begin{itemize}
\tightlist
\item
  \texttt{CO2\_income\_elasticity}: Income elasticity - price of CO2.
\item
  \texttt{sum\_qcpc\_post\_recycle}: Sum of quantiles consumption per capita after abatement, damages and revenue recycling (thousand USD2017 per (capita per quantile) per year).
\item
  \texttt{CPC\_post}: National consumption per capita after recycling, in thousands of USD2017 per person per year.
\item
  \texttt{gini\_cons}: Gini index of country consumption (\%).
\end{itemize}

\subsubsection{Time, country and quantile
dependent}\label{time-country-and-quantile-dependent}

\begin{itemize}
\tightlist
\item
  \texttt{abatement\_cost\_dist}, \texttt{carbon\_tax\_dist},
  \texttt{damage\_dist}: Shares of the distribution of abatement costs,
  CO2 tax burden and climate damage per quantile.
\item
  \texttt{qcpc\_base}: Consumption per quantile per capita before damage, before
  abatement cost, before tax (thousands of USD2017 per person per year).
\item
  \texttt{qcpc\_post\_damage\_abatement}: Consumption per quantile per capita after damage, after
  abatement (thousands of
  USD2017 per person per year).
\item
  \texttt{qcpc\_post\_tax}: Consumption per quantile per capita after
  subtraction of the carbon tax (thousands USD2017 per person per year).
\item
  \texttt{qcpc\_post\_recycle}: Consumption per quantile per capita after
  recycling the carbon tax to each quantile (thousands USD2017 per
  person per year).
\item
  \texttt{qc\_share}: Proportion of consumption per quantile per capita
  (\%).
\end{itemize}

\subsubsection{\texorpdfstring{Time and \texttt{regionwpp}
dependent}{Time and regionwpp dependent}}\label{time-and-regionwpp-dependent}

\begin{itemize}
\tightlist
\item
  \texttt{CPC\_post\_rwpp}: Regional per capita consumption after
  recycling, in thousands of USD2017 per person per year.
\item
  \texttt{gini\_cons\_rwpp}: Gini index of regional consumption (\%).
\end{itemize}

\subsection{Equation}\label{equation-1}

The \texttt{run\_timestep} function calculates various economic
indicators for each time step, country and income quantile.


\subsubsection{Temporary variables for global population and by quantile (in
thousands)}\label{population-by-quantile-in-thousands}

\subsubsection{Income elasticity - CO2
price}\label{income-elasticity---co2-price}

\begin{equation}
  \text{CO2\_income\_elasticity}[t,c] = \text{elasticity\_intercept} + \text{elasticity\_slope} \times log(GDP)
\end{equation}


where : 
\begin{equation}
 GDP = 
 \begin{cases}
 pc\_gdp[t,c] & \text{if  pc\_gdp[t,c]} \in \text{[min\_study\_gdp, max\_study\_gdp]} \\
 min\_study\_gdp & \text{if  pc\_gdp[t,c] < min\_study\_gdp} \\
 max\_study\_gdp & \text{if  pc\_gdp[t,c] > max\_study\_gdp}
 \end{cases} 
\end{equation}


\subsubsection{Distribution of Tax Burdens and Mitigation
Costs}\label{distribution-of-tax-burdens-and-mitigation-costs}

The distributions of the shares of the carbon tax burden
(\texttt{carbon\_tax\_dist}), mitigation costs
(\texttt{abatement\_cost\_dist}), and climate damages
(\texttt{damage\_dist}) per quantile are calculated using the
\texttt{country\_quantile\_distribution} function.

\begin{itemize}
\tightlist
\item 
The carbon tax burden and mitigation costs are equal, calculated with the function \texttt{country\_quantile\_distribution} (see \ref{countryquantiledistribution} for more details).
\begin{multline}
  \left. \begin{array}{l}
  \texttt{carbon\_tax\_dist{[}t,c,q{]}} \\
  \texttt{abatement\_cost\_dist{[}t,c,q{]}}
  \end{array} \right\}
  = \\
  \texttt{country\_quantile\_distribution} \Bigl( \texttt{CO2\_income\_elasticity{[}t,c{]}},\\
  \texttt{quantile\_income\_shares{[}t,c,q{]},\ nb\_quantile} \Bigr)
  \end{multline}

\item Climate damages are calculated by replacing the income-price elasticity of CO2 by the damage-income elasticity
  \texttt{damage\_elasticity} in the above function.
\begin{multline}
  \texttt{damage\_dist{[}t,c,q{]}} = \\
  \texttt{country\_quantile\_distribution} \Bigl( \texttt{damage\_elasticity},\\
  \texttt{quantile\_income\_shares{[}t,c,q{]},\ nb\_quantile)} \Bigr)
\end{multline}

\end{itemize}

\subsubsection{Consumption per Quantile}\label{consumption-per-quantile}

\begin{multline}
  \text{qcpc\_base}[t,c,q] = \text{nb\_quantile} \times \text{CPC}[t,c] \times \text{quantile\_income\_shares}[t,c,q]  \\
 \times \frac{1 + \text{LOCAL\_DAMFRAC\_KW}[t,c]}{1 - \text{ABATEFRAC}[t,c]} 
\end{multline}


\begin{multline}
  \text{qcpc\_post\_damage\_abatement}[t,c,q] = max \biggl[\text{qcpc\_base}[t,c,q] - \text{nb\_quantile} \times \text{CPC}[t,c] \\
 \times \biggl(\text{LOCAL\_DAMFRAC\_KW}[t,c] \times \text{damage\_dist}[t,c,q] \\
 - \frac{1 + \text{LOCAL\_DAMFRAC\_KW[t,c]}}{1 - \text{ABATEFRAC}[t,c]} \times \text{ABATEFRAC}[t,c] \times \text{abatement\_cost\_dist}[t,c,q]\biggr), 10^{-8} \biggr] 
\end{multline}


\begin{multline}
  \text{qcpc\_post\_tax}[t,c,q] = \text{qcpc\_post\_damage\_abatement}[t,c,q] \\
 - (\text{nb\_quantile} \times \text{tax\_pc\_revenue}[t,c] \times \text{carbon\_tax\_dist}[t,c,q]) 
\end{multline}


\begin{itemize}
\item
  If the revenues are recycled, then consumption per post-tax quantile
  is increased by the product of the total per capita revenue from the
  carbon tax and the share of its recycling to each quantile:
  \begin{multline}
    \text{qcpc\_post\_recycle}[t,c,q] = \text{qcpc\_post\_tax}[t,c,q] \\
+ (\text{nb\_quantile} \times \text{country\_pc\_dividend}[t,c] \times \text{recycle\_share}[c,q])
  \end{multline}

\item
  If revenues are not recycled, then consumption per post-tax quantile
  is increased by the product of the carbon tax revenue per capita from
  each country's emissions and its initial tax burden:
  \begin{multline}
    \text{qcpc\_post\_recycle}[t,c,q] = \text{qcpc\_post\_tax}[t,c,q] \\
 + (\text{nb\_quantile} \times \text{tax\_pc\_revenue}[t,c] \times \text{carbon\_tax\_dist}[t,c,q]) 
  \end{multline}

  \subsubsection{Calculation of per capita consumption}\label{calculation-per-capita-consumption}

\item
Sum of country quantiles' consumption per capita after abatement, damages and revenue recycling (thousand USD2017 per (capita per quantile) per year)
  \begin{equation}
    \text{sum\_qcpc\_post\_recycle}[t,c] = \sum_q \text{qcpc\_post\_recycle}[t,c,q]
\end{equation}

\item
  Per capita consumption after recycling:
  \begin{equation}
    \text{CPC\_post}[t,c] = \frac{\text{sum\_qcpc\_post\_recycle}[t,c]}{\text{nb\_quantile}} 
\end{equation}

\item
  Share of country consumption by quantile (\%):
  \begin{equation}
 \text{qc\_share}[t,c,q] = \frac{\text{qcpc\_post\_recycle}[t,c,q]}{\text{sum\_qcpc\_post\_recycle}[t,c]} \times 100
\end{equation}


\subsubsection{Calculation of the Gini Index and the Poverty Rate}\label{calculation-of-the-gini-index-and-the-poverty-rate}
\end{itemize}

The Gini index (\texttt{gini\_cons}) and the poverty rate
(\texttt{poverty\_rate\_cons}) are calculated using the quantile
distribution of consumption.

\begin{itemize}
\item 
  The \textbf{Gini Index} is calculated using the \href{https://docs.juliahub.com/General/Inequality/stable/library/#Inequality.gini}{Gini function in Julia} :
  \begin{equation}
    \texttt{gini\_cons[t,c]} = \texttt{gini} \Bigl( \texttt{convert(Vector\{Real\}}, \texttt{qc\_share[t,c,:])} \Bigr) \times 100
  \end{equation}


\subsubsection{Regional and global calculations}\label{regional-and-global-calculations}
\end{itemize}

Country indices \(c\) and regional indices \(rwpp\) are mapped using
\texttt{mapcrwpp}.

\begin{itemize}
\item
  Regional per capita consumption after recycling:
  \begin{equation}
  CPC\_post\_rwpp[t, rwpp] = \frac{\sum_{c \in rwpp} (l[t, c] \cdot CPC\_post[t, c])}{\sum_{c \in rwpp} l[t, c]}
  \end{equation}
\item
  \texttt{CPC\_post\_global} is calculated as the weighted sum of
  countries' total consumption after recycling, divided by the world
  population:
  \begin{equation}
  CPC\_post\_global[t] = \frac{\sum_c (total\_c\_post\_recycle[t, c] \cdot l[t, c] / nb\_quantile)}{\sum_c l[t, c]}
  \end{equation}
\item
  Regional Gini index: 
  \begin{multline} 
    \texttt{gini\_cons\_rwpp[t,\ rwpp]} = \\
    \texttt{gini} \biggl( \texttt{convert} \Bigl( \texttt{Vector\{Real\},} \texttt{vec(qc\_post\_recycle[t,\ country\_indices,\ : ])} \Bigr), \\
    \texttt{convert} \Bigl( \texttt{Vector\{Real\},\ vec(qpop[t,\ country\_indices,\ :])} \Bigr) \biggr) \times 100
    \end{multline}

\item
  Global Gini index: 
  \begin{multline} 
    \texttt{global\_gini\_cons{[}t{]}} = \texttt{gini} \biggl( \texttt{convert} \Bigl( \texttt{Vector\{Real\},} \texttt{vec(qc\_post\_recycle[t,\ :,\ : ])} \Bigr), \\
    \texttt{convert} \Bigl( \texttt{Vector\{Real\},\ vec(qpop[t,\ :,\ :])} \Bigr) \biggr) \times 100
    \end{multline}

\end{itemize}

\section{revenue\_recycle.jl}\label{revenue_recycle.jl}

This component models the recycling of carbon tax revenues across
different countries and quantiles. It is essential for studying
mitigation policies and their impact on reducing poverty and inequality.
\texttt{revenue\_recycle.jl} is used by \texttt{nice\_v2\_module.jl}
and interacts with \texttt{quantile\_distribution.jl} and
\texttt{welfare.jl}.

\subsection{Index}\label{index-2}

\begin{itemize}
\tightlist
\item
  \texttt{country} : Countries included in the model.
\end{itemize}

\subsection{Parameters}\label{parameters-6}

\subsubsection{Time and country
dependent}\label{time-and-country-dependent-12}

\begin{itemize}
\tightlist
\item
  \texttt{Y}: Net production, in millions of USD2017 per year.
\item
  \texttt{country\_carbon\_tax}: Carbon tax by country (\$/tCO2).
\item
  \texttt{LOCAL\_DAMFRAC\_KW}: National damage as a percentage of net
  GDP calculated from national temperatures.
\item
  \texttt{E\_gtco2}: Industrial carbon emissions (GtCO2 per year).
\item
  \texttt{l}: Country population (in thousands).
\item
  \texttt{tax\_pc\_revenue}: Carbon tax revenue per capita from the
  country's emissions, in thousands of USD2017 per person per year.
\end{itemize}

\subsubsection{Country dependent}\label{country-dependent-3}

\begin{itemize}
\tightlist
\item
  \texttt{global\_recycle\_share}: Share of country revenues that are
  recycled globally in the form of international transfers (1 = 100\%).
\end{itemize}

\subsubsection{Independent}\label{Independent-20}

\begin{itemize}
\tightlist
\item
  \texttt{switch\_recycle}: Recycling of carbon tax revenues.
\item
  \texttt{lost\_revenue\_share}: Portion of carbon tax revenue that is
  lost and cannot be recycled (1 = 100\% of revenue lost, 0 = no revenue
  lost).
\item
  \texttt{switch\_recycle}: Boolean for recycling carbon tax revenue.
\item
  \texttt{switch\_scope\_recycle}: Boolean, carbon tax revenues are
  recycled at national (0) or global (1) level.
\item
  \texttt{switch\_global\_pc\_recycle}: Boolean, carbon tax revenues are
  recycled globally on an equal per capita basis (1).
\end{itemize}

\subsection{Variables}\label{variables-7}

\subsubsection{Time-dependent}\label{time-dependent-6}

\begin{itemize}
\tightlist
\item
  \texttt{total\_tax\_revenue}: Total carbon tax revenue (USD2017/year), sum
  of tax revenues in all countries.
\item
  \texttt{total\_tax\_pc\_revenue}: Total carbon tax revenue per person per year
  (in thousands of dollars per person per year), sum of tax revenues in all
  countries per person per year.
\item
  \texttt{global\_revenue}: Carbon tax revenue, derived from the total
  recycled revenue of all countries (in thousand USD2017 per year).
  \item 
  \texttt{revenue\_recycled\_global\_level}: Sum for all countries of \texttt{tax\_revenue} recycled at a \texttt{global\_recycle\_share}. 
  Mesured in thousands of USD2017 per year.
\end{itemize}

\subsubsection{Time and country
dependent}\label{time-and-country-dependent-13}

\begin{itemize}
\tightlist

\item
  \texttt{tax\_revenue}: Carbon tax revenue for a given country, in USD2017 per year.
\item
  \texttt{tax\_pc\_revenue}: Carbon tax revenue per person for a given
  country, in thousands of USD2017 per person per year.
\item
  \texttt{country\_pc\_dividend}: Total fiscal dividends per person,
  including all international monetary transfers, in thousands of
  USD2017 per person per year.
\item
  \texttt{country\_pc\_dividend\_domestic\_transfers}: Fiscal dividends
  per person from domestic redistribution, i.e.~within a country, in
  thousands of USD2017 per person per year.
\item
  \texttt{country\_pc\_dividend\_global\_transfers}: Tax dividends per
  person from international transfers, in thousands of USD2017 per
  person per year.
\end{itemize}

\subsection{Equation}\label{equation-2}


\subsubsection{Carbon tax revenue}\label{carbon-tax-revenue}

\begin{multline}
 \text{tax\_revenue}[t,c] = \left(E\_gtco2[t,c] \times \text{country\_carbon\_tax}[t,c] \times 10^9 \right) \\
 \times \left(1 - \text{lost\_revenue\_share} \right) 
\end{multline}

Emissions are in GtCO2 and \texttt{country\_carbon\_tax} in USD2017/tCO2, hence the
normalisation by \(10^9\).

\begin{equation}
 \text{tax\_pc\_revenue}[t,c] = \frac{tax\_revenue[t,c]}{l[t,c] \times 10^6} 
\end{equation}


\texttt{tax\_revenue} is in USD2017/year and population in thousands of
inhabitants, hence the normalisation by \(10^{-6}\).

\begin{equation}
 \text{total\_tax\_revenue}[t] = \sum_c \text{tax\_revenue}[t,c]
\end{equation}

\begin{equation}
 \text{total\_tax\_pc\_revenue}[t] = \frac{\text{total\_tax\_revenue}[t]}{\sum_c l[t,c] \times 10^6}
\end{equation}

  \begin{equation}
  \text{global\_revenue}[t] = \sum_c \Bigl( \text{tax\_revenue}[t,c] \times \text{global\_recycle\_share}[c] \times \text{switch\_scope\_recycle} \Bigr)
  \end{equation}

  \begin{equation}
    \text{global\_pc\_revenue}[t] = \frac{\text{global\_revenue}[t]}{\sum_c(l[t,c]) \times 10^6}
  \end{equation}


\subsubsection{Dividends according to Boolean
values}\label{dividends-according-to-boolean-values}

\paragraph{\texorpdfstring{\texttt{switch\_recycle\ =\ 1}}{switch\_recycle = 1}}\label{switch_recycle-1}

No income recycling implies no dividends. 

\begin{equation}
  \left. \begin{array}{l}
  \text{country\_pc\_dividend\_domestic\_transfers}[t,c] \\
  \text{country\_pc\_dividend\_global\_transfers}[t,c] \\
  \end{array} \right\}
  = 0
  \end{equation}

\paragraph{\texorpdfstring{\texttt{switch\_recycle==1},
\texttt{switch\_scope\_recycle==0}}{switch\_recycle==1, switch\_scope\_recycle==0}}\label{switch_recycle1-switch_scope_recycle0}

Revenues from the carbon tax are recycled at a national level:
\begin{equation}
\text{country\_pc\_dividend\_domestic\_transfers}[t,c] = \frac{tax\_revenue[t,c]}{l[t,c] \times 10^6}
\end{equation}

\begin{equation}
 \text{country\_pc\_dividend\_global\_transfers}[t,c] = 0
\end{equation}


\paragraph{\texorpdfstring{\texttt{switch\_recycle==1},
\texttt{switch\_scope\_recycle==1}}{switch\_recycle==1, switch\_scope\_recycle==1}}\label{switch_recycle1-switch_scope_recycle1}

Carbon tax revenues are recycled at a global level:
\begin{equation}
\text{country\_pc\_dividend\_domestic\_transfers}[t,c] = (1 - \text{global\_recycle\_share}[c]) \times \frac{tax\_revenue[t,c]}{l[t,c] \times 10^6} 
\end{equation}

\begin{equation}
\text{revenue\_recycled\_global\_level}[t] = \sum_c(\text{tax\_revenue}[t,c] \times \text{global\_recycle\_share}[c]) \times 10^{-3}
\end{equation}


\paragraph{\texorpdfstring{\textbf{Booleans change the way dividends are calculated}.}{Booleans change the way dividends are calculated.}}\label{booleans-change-the-way-dividends-are-calculated.}

\begin{itemize}
\item
  \texttt{switch\_pc\_global\_recycle\ =\ 1}; Carbon tax revenues are
  recycled globally on an equal per capita basis.
  \begin{equation}
 \text{country\_pc\_dividend\_global\_transfers}[t,c] = \frac{\text{revenue\_recycled\_global\_level}[t]}{\sum_c(l[t,c]) \times 10^3} 
\end{equation}


   Otherwise:
   \begin{equation}
   \text{country\_pc\_dividend\_global\_transfers}[t,c] = 0
   \end{equation}

\end{itemize}

\subsubsection{Total dividends}\label{total-dividends}

Total dividends per person are written as:
\begin{multline}
 \text{country\_pc\_dividend}[t,c] = \text{country\_pc\_dividend\_domestic\_transfers}[t,c] \\ 
 + \text{country\_pc\_dividend\_global\_transfers}[t,c]
\end{multline}


\section{welfare.jl}\label{welfare.jl}

This file calculates economic welfare and consumption equivalent to
equitably distributed welfare (EDE) as a function of inequality aversion
(\(\eta\)). \texttt{welfare.jl} is loaded by
\texttt{nice\_v2\_module.jl} and uses data provided by
\texttt{revenue\_recycle.jl} and \texttt{quantile\_distribution.jl}.

\subsection{Indexes}\label{indexes-4}

\begin{itemize}
\tightlist
\item
  \texttt{country}, \texttt{regionwpp}, \texttt{quantile} : Indexes
  representing countries, regions and income quantiles respectively.
\end{itemize}

\subsection{Parameters}\label{parameters-7}

\begin{itemize}
\tightlist
\item
  \texttt{qcpc\_post\_recycle}: Per capita quantile consumption after
  redistribution of the recycling tax (in thousands of USD2017 per
  person per year). It depends on time, country and quantile.
\item
  \(\eta\): Inequality aversion.
\item
  \texttt{nb\_quantile}: Number of quantiles.
\item
  \texttt{l}: Population (in thousands). This parameter is time and country dependent.
\item
  \texttt{mapcrwpp}: Correspondence of the country index with the WPP
  regions index (country dependent).
\end{itemize}

\subsection{Variables}\label{variables-8}

\begin{itemize}
\tightlist
\item
  \texttt{cons\_EDE\_country}, \texttt{cons\_EDE\_rwpp},
  \texttt{cons\_EDE\_global}: Consumption equivalent to equitably
  distributed well-being (in thousands of USD2017 per person per year)
  for countries, regions and globally.
\item
  \texttt{welfare\_country}, \texttt{welfare\_rwpp},
  \texttt{welfare\_global}: Welfare for countries, regions and globally.
\end{itemize}

\subsection{Equations}\label{equations-5}

\subsubsection{\texorpdfstring{Case where
$\eta \neq 1$}{Case where \textbackslash eta \textbackslash neq 1}}\label{case-where-eta-neq-1}

For each country (\texttt{c}), EDE consumption and welfare are
calculated as follows:

\paragraph{EDE consumption by
country}\label{ede-consumption-by-country}

\begin{equation}
  cons\_EDE\_country[t,c] = \left( \frac{1}{nb\_quantile} \sum_q (qcpc\_post\_recycle[t,c,q]^{(1-\eta)} ) \right)^{\frac{1}{(1-\eta)}}
\end{equation}

\paragraph{EDE consumption by
region}\label{ede-consumption-by-region}

For each region, the country indices \(c\) are mapped using
\texttt{mapcrwpp}.

\begin{equation}
  cons\_EDE\_rwpp[t,rwpp] = \left( \frac{\sum_{c \in rwpp} (l[t,c] \cdot cons\_EDE\_country[t,c]^{(1-\eta)})}{\sum_{c \in rwpp} l[t,c]} \right)^{\frac{1}{(1-\eta)}}
\end{equation}

\paragraph{Overall EDE consumption}\label{overall-ede-consumption}

\begin{equation}
 cons\_EDE\_global[t] = \left( \frac{\sum_c (l[t,c] \cdot cons\_EDE\_country[t,c]^{(1-\eta)})}{\sum_c l[t,c]} \right)^{\frac{1}{(1-\eta)}} 
\end{equation}


\paragraph{Welfare by country}\label{welfare-by-country}

\begin{equation}
  welfare\_country[t,c] = \left( \frac{l[t,c]}{nb\_quantile} \right) \sum_q \left( \frac{qcpc\_post\_recycle[t,c,q]^{(1-\eta)}}{(1-\eta)} \right)
\end{equation}

\paragraph{Welfare by region}\label{welfare-by-region}

\begin{equation} 
  welfare\_rwpp[t,rwpp] = \sum_{c \in rwpp}welfare\_country[t,c]
\end{equation}

\paragraph{Global welfare}\label{global-welfare}

\begin{equation}
  welfare\_global[t] = \sum_c (welfare\_country[t,c])
\end{equation}

\subsubsection{\texorpdfstring{Case where
\(\eta = 1\)}{Case where \textbackslash eta = 1}}\label{case-where-eta-1}

For each country (\texttt{c}), EDE consumption and welfare are
calculated using the logarithm :

\paragraph{EDE consumption by
country}\label{ede-consumption-by-country-1}

\begin{equation}
  cons\_EDE\_country[t,c] = \exp \left( \frac{1}{nb\_quantile} \sum_q \log(qcpc\_post\_recycle[t,c,q]) \right)
\end{equation}

\paragraph{EDE consumption by
region}\label{ede-consumption-by-region-1}

\begin{equation}
  cons\_EDE\_rwpp[t,rwpp] = \exp \left( \frac{\sum_{c \in rwpp} l[t,c] \cdot \log(cons\_EDE\_country[t,c])}{\sum_{c \in rwpp} l[t,c]} \right)
\end{equation}

\paragraph{Total EDE consumption}\label{total-ede-consumption}

\begin{equation}
  cons\_EDE\_country[t] = \exp\left( \frac{\sum_c l[t,c] \cdot \log(cons\_ED\_country[t,c])}{\sum_c l[t,c]} \right)
\end{equation}

\paragraph{Welfare by country}\label{welfare-by-country-1}

\begin{equation}
  welfare\_country[t,c] = \left( \frac{l[t,c]}{nb\_quantile} \right) \sum_q \log(qcpc\_post\_recycle[t,c,q])
\end{equation}

\paragraph{Welfare by region}\label{welfare-by-region-1}

\begin{equation}
  welfare\_rwpp[t,rwpp] = \sum_{c \in rwpp}welfare\_country[t,c]
\end{equation}

\paragraph{Global welfare}\label{global-welfare-1}

\begin{equation}
  welfare\_global[t] = \sum_c (welfare\_country[t,c])
\end{equation}

\section{helper\_functions.jl}\label{helper_functions.jl}

\subsection{Functions}\label{functions}

\subsubsection{\texorpdfstring{\texttt{country\_quantile\_distribution}}{Country\_quantile\_distribution}} \label{countryquantiledistribution}

\paragraph{Objective}\label{objective}

The \texttt{country\_quantile\_distribution} function calculates the
distribution of quantile shares for a given country, based on a given
income elasticity. This function is useful for analysing the impact of
climate damage, $CO_2$ mitigation costs, or $CO_2$ related tax burdens on
different segments of the population within a country.

\paragraph{Arguments of the
function}\label{arguments-of-the-function}

\begin{itemize}
\tightlist
\item
  \texttt{elasticity}: Income elasticity with respect to climate damage,
  $CO_2$ mitigation costs, $CO_2$ tax burdens, etc. This must be a real
  number.
\item
  \texttt{income\_shares}: An array of income shares by quantile (where
  the rows represent the countries and the columns the quantiles)
\item
  \texttt{nb\_quantile}: Number of quantiles, specified as an integer.
\end{itemize}

\paragraph{Process}\label{process}

\begin{enumerate}
\def\labelenumi{\arabic{enumi}.}
\tightlist
\item
  \textbf{Application of elasticity}: For a given country, income elasticity is applied to
  income shares per quantile by raising each income share to the power
  of the elasticity.
  \begin{equation} 
    \text{scaled\_shares}[t,c,q] = \text{income\_shares}^{\text{elasticity}}[t,c,q] 
  \end{equation}
\item
  \textbf{Allocate an empty array}: An empty array \texttt{updated\_quantile\_distribution} is allocated to store
  the distribution across quantiles resulting from the application of
  the elasticity.
\item
  \textbf{Calculation of updated distributions}: The function iterates
  over each quantile to calculate the updated distribution. For each
  quantile, the updated share is calculated by dividing the scaled
  income share for that quantile by the total sum of scaled income
  shares for all quantiles.

  \begin{equation} 
    \text{updated\_quantile\_distribution}[t,c,q] = \frac{\text{scaled\_shares}[t,c,q]}{\sum_{q=1}^{\text{nb\_quantile}} \text{scaled\_shares}[t,c,qq]} 
  \end{equation}
\end{enumerate}

\paragraph{Return}\label{return-1}

The function returns the array \texttt{updated\_quantile\_distribution},
which contains the updated quantile share distribution for each country,
reflecting the impact of income elasticity on the distribution of
economic costs or damages.

\subsubsection{\texorpdfstring{\texttt{linear\_tax\_trajectory}}{Linear\_tax\_trajectory}}\label{linear_tax_trajectory}

This function calculates a linear carbon tax trajectory. It assumes that
a carbon tax is \$0 in the first period and increases linearly until the
end of the specified period.

\paragraph{Function arguments}\label{function-arguments-1}

\begin{itemize}
\tightlist
\item
  \texttt{tax\_start\_value} (Real): The initial value of the carbon tax
  (in 2017 US dollars per tonne of CO2).
\item
  \texttt{increase\_value} (Real): The annual tax increase value. By
  default, it is equal to \texttt{tax\_start\_value}, which means that
  the tax increases by its initial value each year.
\item
  \texttt{year\_tax\_start} (Int64): The first year of the tax
  trajectory. The tax starts at 0 in this year and increases to
  \texttt{tax\_start\_value} the following year.
\item
  \texttt{year\_tax\_end} (Int64): The last year for which to calculate
  the tax.
\item
  \texttt{year\_step} (Int64): The step in years between two tax values.
  The default is 1, indicating an annual increase.
\item
  \texttt{year\_model\_end} (Int64): The end of the model. If it is less
  than \texttt{year\_tax\_end}, the last tax value is repeated up to
  this year. By default, it is set to 2300.
\end{itemize}

\paragraph{Description}\label{description}

The \texttt{linear\_tax\_trajectory} function generates a carbon tax
trajectory which starts at \$0 and increases linearly each year
according to the \texttt{increase\_value} specified, from year
\texttt{year\_tax\_start\ +\ 1} to year \texttt{year\_tax\_end}. After
\texttt{year\_tax\_end}, the tax value remains constant until
\texttt{year\_model\_end} if specified.

\footnotesize
\begin{mdframed}
\begin{Shaded}
\begin{Highlighting}[]
\KeywordTok{function} \FunctionTok{linear\_tax\_trajectory}\NormalTok{(;tax\_start\_value}\OperatorTok{::}\DataTypeTok{Real}\NormalTok{, increase\_value}\OperatorTok{::}\DataTypeTok{Real}\NormalTok{=tax\_start\_value,} \\
 \NormalTok{year\_tax\_start}\OperatorTok{::}\DataTypeTok{Int64}\NormalTok{, year\_tax\_end}\OperatorTok{::}\DataTypeTok{Int64}\NormalTok{, year\_step}\OperatorTok{::}\DataTypeTok{Int64}\NormalTok{=}\FloatTok{1}\NormalTok{, year\_model\_end}\OperatorTok{::}\DataTypeTok{Int64}\NormalTok{=}\FloatTok{2300}\NormalTok{)}

\NormalTok{for t }\KeywordTok{in}\NormalTok{ [year\_tax\_start}\OperatorTok{+}\FloatTok{1}\OperatorTok{:}\NormalTok{year\_step}\OperatorTok{:}\NormalTok{year\_tax\_end]}
\NormalTok{    tax\_values }\OperatorTok{=}\NormalTok{ tax\_start\_value }\OperatorTok{+}\NormalTok{ increase\_value }\OperatorTok{*}\NormalTok{ (}\FunctionTok{t{-}}\NormalTok{(year\_tax\_start}\OperatorTok{+}\FloatTok{1}\NormalTok{)) }

\NormalTok{    full\_tax\_path }\OperatorTok{=}\NormalTok{ [}\FloatTok{0}\NormalTok{; tax\_values; }\FunctionTok{fill}\NormalTok{(tax\_values[}\KeywordTok{end}\NormalTok{], year\_model\_end}\OperatorTok{{ -}}\NormalTok{ year\_tax\_end)]}

    \ControlFlowTok{return}\NormalTok{ full\_tax\_path}
\KeywordTok{end}
\end{Highlighting}
\end{Shaded}
\end{mdframed}
\normalsize

For $t \in [\text{year\_tax\_start}+1:\text{year\_step}:\text{year\_tax\_end}]$:
\begin{equation}
  \text{tax\_values}[t] = [\text{tax\_start\_value} + \text{increase\_value} \times (t - (\text{year\_tax\_start} + 1) )
\end{equation}

\begin{equation}
  \text{full\_tax\_path} = [0; \text{tax\_values}; \text{fill}(\text{tax\_values[end]}, \text{year\_model\_end} - \text{year\_tax\_end})]
\end{equation}

\paragraph{Return}\label{return-2}

The function returns a vector containing the complete trajectory of the
carbon tax, from the start year to the end year of the model.

\paragraph{Example of use}\label{example-of-use}

\small
\begin{Shaded}
\begin{Highlighting}[]
\NormalTok{linear\_tax\_path }\OperatorTok{=} \FunctionTok{linear\_tax\_trajectory}\NormalTok{(tax\_start\_value}\OperatorTok{=}\FloatTok{10}\NormalTok{, increase\_value}\OperatorTok{=}\FloatTok{5}\NormalTok{,} \\ 
\NormalTok{year\_tax\_start}\OperatorTok{=}\FloatTok{2020}\NormalTok{, year\_tax\_end}\OperatorTok{=}\FloatTok{2050}\NormalTok{, year\_step}\OperatorTok{=}\FloatTok{1}\NormalTok{, year\_model\_end}\OperatorTok{=}\FloatTok{2100}\NormalTok{)}
\end{Highlighting}
\end{Shaded}
\normalsize

\subsubsection{\texorpdfstring{\texttt{save\_nice2020\_results}}{Save\_nice2020\_results}}\label{save_nicev2_results}

\paragraph{Description}\label{description-2}

This function creates a directory of files to store the results of a
model, dividing the model output by global, regional and quantile
levels. A different sub-folder name is used for each type of revenue recycle (careful: revenue recycling type must be set with the use of the parameter switches).

\paragraph{Arguments}\label{arguments-1}

\begin{itemize}
\tightlist
\item
  \texttt{m::Model} : A version of the NICE model.
\item
\texttt{output\_directory::String} : The path to the directory where the model
  results will be saved.
\item
  \texttt{recycling\_revenue::Bool=true} : Indicates whether CO2 tax
  revenues are recycled (\texttt{true}) or not (\texttt{false}).
\item
  \texttt{recycling\_type::Int64=0} : The type of revenue recycling.
\item
  \texttt{result\_year\_end::Int64=2100} : The end year for the results.
\end{itemize}

\paragraph{Process}\label{process-1}

The function begins by creating sub-directories to store the results,
with and without revenue recycling. The type of revenue recycling is
determined by the \texttt{recycling\_type} argument, which influences
the name of the sub-folder.

The function then saves different types of results in the appropriate
directories:

\begin{itemize}
\tightlist
\item
  \textbf{Global Output}: Includes files such as global temperature,
  global gross GDP, CO2 emissions, etc.
\item
  \textbf{Regional Output}: Includes files such as consumption per
  capita per region, net GDP per capita per region, etc.
\item
  \textbf{Output by Country}: Includes files such as gross GDP per
  country, consumption per country, population per country, etc.
\item
  \textbf{Output by Quantile}: Includes files such as CO2 tax
  distribution, per capita consumption, consumption after damage and
  abatement, etc.
\end{itemize}

\paragraph{Example of use}\label{example-of-use-2}

\begin{Shaded}
\begin{Highlighting}[]
\NormalTok{m }\OperatorTok{=} \FunctionTok{load\_modele\_nice}\NormalTok{() }\CommentTok{\# Hypothetical function to load a NICE model}
\NormalTok{output\_directory }\OperatorTok{=} \StringTok{"path/to/folder/results"}
\FunctionTok{save\_nicev2\_results}\NormalTok{(m, output\_directory, revenue\_recycling}\OperatorTok{=}\ConstantTok{true}\NormalTok{, recycling\_type}\OperatorTok{=}\FloatTok{1}\NormalTok{)}
\end{Highlighting}
\end{Shaded}

\subsubsection{\texorpdfstring{\texttt{get\_global\_mitigation}}{Get\_global\_mitigation}}\label{get_global_mitigation}

This function calculates the global CO2 mitigation rate for a given
model with a specific mitigation policy, compared with a reference
scenario with no CO2 policy (business as usual, BAU).

\paragraph{Arguments}\label{arguments}

\begin{itemize}
\tightlist
\item
  \texttt{m\_policy} (\texttt{Model}): A model representing the economy
  or environmental system with a CO2 mitigation policy applied.
\item
  \texttt{m\_bau} (\texttt{Model}): A model representing the reference
  scenario (business as usual) without a CO2 mitigation policy.
\end{itemize}

\paragraph{Description}\label{description-1}

The function performs the following operations:
\begin{enumerate}
\def\labelenumi{\arabic{enumi}.}
\tightlist
\item
  \textbf{Calculates the global industrial CO2 emissions} for the
  \texttt{m\_policy} model which includes a specific mitigation policy.
  \begin{equation}
    \text{global\_emissions\_policy} = \sum (\text{m\_policy}[:\text{emissions}, :\text{E\_Global\_gtco2}], \text{dims}=2)
    \end{equation}
\item
  \textbf{Calculation of global CO2 emissions without policy (BAU)}:
  Calculates global industrial CO2 emissions for the \texttt{m\_bau}
  model, which represents a scenario without a CO2 emissions mitigation
  policy.
  \begin{equation}
    \text{global\_emissions\_base} = \sum (\text{m\_bau}[:\text{emissions}, :\text{E\_Global\_gtco2}], \text{dims}=2)
    \end{equation}
\item
  \textbf{Calculation of the mitigation rate}: Determines the global
  mitigation rate by comparing the emissions with mitigation policy to
  the emissions of the BAU scenario.
  \begin{equation}
    \text{global\_mitigation\_rates} = \frac{\text{global\_emissions\_base} - \text{global\_emissions\_policy}}{\text{global\_emissions\_base}}
    \end{equation}
\end{enumerate}

\paragraph{Return}\label{return-3}

\begin{itemize}
\tightlist
\item
  \texttt{global\_mitigation\_rates}: A vector containing global CO2
  mitigation rates, calculated as the relative reduction in emissions
  compared with the BAU scenario.
\end{itemize}

\paragraph{Example of use}\label{example-of-use-1}

\begin{Shaded}
\begin{Highlighting}[]
\CommentTok{\# Creating the m\_policy and m\_bau models}
\NormalTok{m\_policy }\OperatorTok{=} \FunctionTok{Model}\NormalTok{(}\OperatorTok{...}\NormalTok{) }\CommentTok{\# Model with mitigation policy}
\NormalTok{m\_bau }\OperatorTok{=} \FunctionTok{Model}\NormalTok{(}\OperatorTok{...}\NormalTok{) }\CommentTok{\# BAU model}

\CommentTok{\# Calculation of global mitigation rate}
\NormalTok{mitigation\_rates }\OperatorTok{=} \FunctionTok{get\_global\_mitigation}\NormalTok{(m\_policy, m\_bau)}
\end{Highlighting}
\end{Shaded}

% \section{nice\_v2\_function.jl}\label{nice_v2_function.jl}
%
% This file defines the \texttt{create\_nice\_v2} function, which creates
% a version of the NICE 2020 simulated model by coupling the FAIRv2.0
% model to a simple damage function. It loads economic data, helper
% functions and additional components from the Mimi model. Components
% loaded include \texttt{components} (the initialization data),
% \texttt{pattern\_scale.jl}, \texttt{revenue\_recycle.jl}, but neither
% \texttt{welfare.jl} nor \texttt{quantile\_distribution.jl}.
%
% It has an equivalent role to the \texttt{nice\_v2\_module.jl} file.

\section{nice2020\_module.jl}\label{nice_v2_module.jl}

This file defines the \texttt{create\_nice2020} function, which creates
a versison of the NICE 2020 simulated model by coupling the FAIRv2.0
model to a simple damage function.

\subsection{Loading packages and data}\label{loading-packages-and-data}

The module uses several Julia packages for its operations:

\begin{itemize}
\tightlist
\item
  \texttt{Mimi}: A framework for developing and analysing AMIs.
\item
  \texttt{MimiFAIRv2}: A specific implementation of the FAIR model
  within the Mimi framework.
\item
  \texttt{Statistics}: Provides basic statistical functions.
\item
  \texttt{Random}: Functions for generating random numbers.
\item
  \texttt{Distributions}: A package for defining and working with
  probabilistic distributions.
\item
  \texttt{Inequality}: Used to analyse inequalities in data or model
  results.
\end{itemize}

Economic data is loaded from an external \texttt{parameters.jl} file,
located in the \texttt{data} folder relative to the module.

\subsection{Mimi Model Auxiliary Functions and
Components}\label{mimi-model-auxiliary-functions-and-components}

The module includes auxiliary functions from the
\texttt{helper\_functions.jl} file. The following model components are
loaded from the \texttt{components} folder:

\begin{itemize}
\item
  \texttt{gross\_economy.jl}: Defines how the gross economy works before
  abatement costs or damages are taken into account.
\item
  \texttt{abatement.jl}: Manages emission abatement costs and
  strategies.
\item
  \texttt{emissions.jl}: Calculates emissions based on various economic
  and abatement factors.
\item
  \texttt{pattern\_scale.jl} : Compute local temperature values.
\item
  \texttt{damages.jl}: Estimates the economic damage caused by climate
  change.
\item
  net\_economy.jl: Calculates the net savings after taking into account
  damage and abatement costs.
\item
  \texttt{revenue\_recycle.jl}: Processes the recycling of revenues
  generated by taxes or emission permits.
\item
  \texttt{quantile\_reycle.jl} : Analyses the distribution of
  impacts or costs across different quantiles of the population.
\item
  \texttt{welfare.jl}: Evaluates general welfare, taking into account
  economic and environmental aspects.
\end{itemize}

\subsection{\texorpdfstring{\texttt{create\_nice2020()}
function}{create\_nice2020() function}}\label{create_nice_v2-function}

The function is based on the Mimi-FAIRv2 model with an emissions and
radiative forcing scenario defined according to SSP2-45, between 2020
and 2300. 20 World Population Prospects
and 10 quantiles are created, \texttt{c} is defined as the symbol
representing the countries (\texttt{countries}).

The emissions, abatement and gross economy components are added before
the FAIR carbon cycle, and this model (FAIR) is coupled with scaling,
regional damages, net production, income recycling, quantile recycling
and welfare valuation components, in that order.

By default, tax revenues are not recycled (i.e., they are returned to each quantile). We define the mapping of countries to the WPP regions and the number of quantiles that will be used.

\subsection{Initial conditions}\label{initial-conditions}
Initial conditions are updated for different
environmental cycles in the FAIR model. Each block of code relates to a
specific cycle and updates the parameters associated with that cycle.

\subsubsection{\texorpdfstring{The \texttt{update\_param!}
function}{The update\_param! function}}\label{the-update_param-function}

An explainer for the Mimi framework syntax for setting and updating parameters can be found at \url{https://www.mimiframework.org/Mimi.jl/stable/howto/howto_5/}.

The \texttt{update\_param!} function is used for setting unshared model parameters.
It generally takes four arguments:

\begin{enumerate}
\def\labelenumi{\arabic{enumi}.}
\tightlist
\item
  \texttt{m}: The model to which the parameter belongs.
\item
  The name of the cycle for which the parameter is being updated.
\item
  The name of the parameter to be updated.
\item
  The new parameter value.
\end{enumerate}

\subsubsection{Aerosol cycle and
temperature}\label{aerosol-cycle-and-temperature}

\begin{Shaded}
\begin{Highlighting}[]
\FunctionTok{update\_param!}\NormalTok{(m, }\OperatorTok{:}\NormalTok{aerosol\_plus\_cycles, }\OperatorTok{:}\NormalTok{aerosol\_plus\_0, init\_aerosol[!,}\OperatorTok{:}\NormalTok{concentration])}

\FunctionTok{update\_param!}\NormalTok{(m, }\OperatorTok{:}\NormalTok{aerosol\_plus\_cycles, }\OperatorTok{:}\NormalTok{R0\_aerosol\_plus, } \\
\FunctionTok{                     Matrix}\NormalTok{(init\_aerosol[!, [}\OperatorTok{:}\NormalTok{R1, }\OperatorTok{:}\NormalTok{R2, }\OperatorTok{:}\NormalTok{R3, }\OperatorTok{:}\NormalTok{R4]]))}

\FunctionTok{update\_param!}\NormalTok{(m, }\OperatorTok{:}\NormalTok{aerosol\_plus\_cycles, }\OperatorTok{:}\NormalTok{GU\_aerosol\_plus\_0, init\_aerosol[!,}\OperatorTok{:}\NormalTok{GU])}
\end{Highlighting}
\end{Shaded}

Initial concentration (\texttt{aerosol\_plus\_0}), radiative response
(\texttt{R0\_aerosol\_plus}), and radiative forcing
(\texttt{GU\_aerosol\_plus\_0}) of aerosols are updated. The same is
done for methane (CH4), carbon dioxide (CO2), fluorinated gases,
substances controlled by the Montreal Protocol, and nitrous oxide (N2O).
Finally, the initial temperature is defined.

\subsubsection{Gross economy}\label{gross-economy}

We connect the population parameter \texttt{l} to the gross economy, and
do the same with total factor productivity, the rate of capital
depreciation, initial capital \texttt{k0} and the share of income
allocated to capital (here \texttt{share} is set to \texttt{0.3}).

\subsection{\texorpdfstring{Abatement :
\texttt{abatement.jl}}{Abatement : abatement.jl}}\label{abatement-abatement.jl}

\subsubsection{\texorpdfstring{\texttt{set\_param!}
function}{set\_param! function}}\label{set_param-function}

The \texttt{set\_param!} function is used for defining or modifying the value
of a specific parameter for a given component of the model.

\begin{itemize}
\tightlist
\item
  The mitigation control regime is defined (1: Global carbon tax, 2:
  Carbon tax per country, 3: Mitigation rate per country), which is set
  to 3 by default.
\item
  Set the global carbon tax (\texttt{global\_carbon\_tax}) and the
  reference carbon tax (\texttt{reference\_carbon\_tax}) to zero for
  each time period.
\item
  The USA is defined as the reference country.
\item
  Initialise \(\mu\) to 0 in order to prepare the model to receive
  specific mitigation rate values for each country and time period.
\item
  Set \(\theta_2 = 2.6\) and \(\eta = 1.5\).
\item
  The cost of the CO2 removal technology over time is specified with
  \texttt{full\_pbacktime}.
\item
  The values of the emission rate \(\sigma\) and the savings rate \(s\)
  are set.
\item
  The population parameter \texttt{l} is connected to the mitigation
  component, so that the value of \texttt{l} is the same throughout the
  model.
\end{itemize}

\subsection{\texorpdfstring{CO2 emissions :
\texttt{emissions.jl}}{CO2 emissions : emissions.jl}}\label{co2-emissions-emissions.jl}

We map countries to WPP regions with \texttt{mapcrwpp}, and update the
emission rate \(\sigma\).

\subsection{\texorpdfstring{Temperature :
\texttt{pattern\_scale.jl}}{Temperature : pattern\_scale.jl}}\label{temperature-pattern_scale.jl}

Update the \(\beta\) temperature dimensioning parameter according to
CMIP projections.

\subsection{\texorpdfstring{Damages :
\texttt{damages.jl}}{Damages : damages.jl}}\label{damages-damages.jl}

The damage coefficients $\beta1$ (= 0.0236), $\beta2$ (= 2), $\beta1\_KW$, and $\beta2\_KW$ are
defined.

\subsection{\texorpdfstring{Net economy :
\texttt{net\_economy.jl}}{Net economy : net\_economy.jl}}\label{net-economy-net_economy.jl}

We map countries to WPP regions with \texttt{mapcrwpp}, update the
savings rate \(s\), and connect the population parameter \texttt{l}.

\subsection{\texorpdfstring{Income recycling :
\texttt{revenue\_recycle.jl}}{Income recycling : revenue\_recycle.jl}}\label{income-recycling-revenue_recycle.jl}

\begin{itemize}
\tightlist
\item
  Assume a damage-income elasticity of 0.85
  (\texttt{damage\_elasticity\_data\ =\ 0.85})
\item
  By default, the proportion of countries' revenues that are recycled
  globally in the form of international transfers is set at 100\% for
  all countries.
\item
  By default, carbon tax revenues are recycled globally
  (\texttt{switch\_scope\_recycle\ =\ 1}), and all other Booleans are
  disabled. The share of revenue lost (\texttt{lost\_revenue\_share}) is
  set to 0.

\end{itemize}

\subsection{\texorpdfstring{Quantiles :
\texttt{quantile\_distribution.jl}}{Quantiles : quantile\_distribution.jl}}\label{quantiles-quantile_distribution.jl}

\begin{itemize}
\tightlist
\item
  The minimum (647 USD2017 per capita) and maximum (48892 USD2017 per
  capita) GDP per capita used in elasticity studies are defined.
\item
  Update the intercept and slope parameters for elasticity, which define
  the relationship between GDP per capita and demand elasticity,
  essential for modelling how economic changes affect consumption.
\item
  The elasticity of damage and income distribution is configured.
\item
  The share of income recycling for a quantile is uniformly defined as
  the inverse of the total number of quantiles.

\item
  The demographic parameters of population and correspondence between
  countries and WPP regions are connected. The Boolean for activating
  income recycling (\texttt{switch\_recycle}) is also connected.
\end{itemize}

\subsection{\texorpdfstring{Welfare :
\texttt{welfare.jl}}{Welfare : welfare.jl}}\label{welfare-welfare.jl}

Similarly, we set certain values (number of quantiles and inequality
aversion \(\eta = 1.5\)) and connect the demographic parameters.

\subsection{Connection between
components}\label{connection-between-components}

The general syntax for creating a connection is :

\begin{Shaded}
\begin{Highlighting}[]
\FunctionTok{connect\_param!}\NormalTok{(model\_name, }\OperatorTok{:}\NormalTok{component\_requiring\_value }\OperatorTok{=\textgreater{}} \OperatorTok{:}\NormalTok{name\_of\_required\_value, } \\
\OperatorTok{:}\NormalTok{component\_calculating\_value }\OperatorTok{=\textgreater{}} \OperatorTok{:}\NormalTok{name\_of\_calculated\_value)}
\end{Highlighting}
\end{Shaded}

\begin{itemize}
\tightlist
\item
  \texttt{model\_name} : The name of the model, in this case m.
\item
  \texttt{:component\_requiring\_value} : The component which requires a
  value from another component.
\item
  \texttt{:name\_of\_required\_value} : The name of the value required
  by the component.
\item
  \texttt{:component\_calculating\_value} : The component which
  calculates the required value.
\item
  name\_of\_calculated\_value` : The name of the value calculated by the
  component.
\end{itemize}

Component connections are: 

\begin{itemize}
\tightlist
\item
  \texttt{I} is calculated by \texttt{neteconomy} and is used by
  \texttt{grosseconomy}.
\item
  \texttt{YGROSS} is calculated by \texttt{grosseconomy} and is used for
  \texttt{abatement}, \texttt{emissions} and \texttt{neteconomy}.
\item
  \(\mu\) is calculated by \texttt{abatement} and used for
  \texttt{emissions}.
\item
  \texttt{E\_Global\_gtc} is calculated by \texttt{emissions}, is
  required under the name \texttt{E\_co2} and is used for
  \texttt{co2\_cycle}.
\item
  \texttt{T} is calculated by \texttt{temperature}, is required under
  the name \texttt{global\_temperature} and is used for
  \texttt{pattern\_scale}.
\item
  \texttt{T} is calculated by \texttt{temperature}, is required under
  the name \texttt{temp\_anomaly} and is used for \texttt{damages}.
\item
  \texttt{local\_temperature} is calculated by \texttt{pattern\_scale},
  is required under the name \texttt{local\_temp\_anomaly} and is used
  for \texttt{damages}.
\item
  \texttt{ABATEFRAC} is calculated by \texttt{abatement} and is used for
  \texttt{neteconomy}.
\item
  \texttt{LOCAL\_DAMFRAC\_KW} is calculated by \texttt{damages} and is
  used for \texttt{neteconomy}, \texttt{revenue\_recycle}, and
  \texttt{quantile\_recycle}.
\item
  \texttt{E\_gtco2} is calculated by \texttt{emissions} and is used for
  \texttt{revenue\_recycle}.
\item
  \texttt{country\_carbon\_tax} is calculated by \texttt{abatement} and
  is used for \texttt{revenue\_cycle}.
\item
  \texttt{Y} is calculated by \texttt{neteconomy} and used for
  \texttt{revenue\_recycle} and \texttt{quantile\_recycle}.
\item
  \texttt{CPC} is calculated by \texttt{neteconomy} and used for
  \texttt{quantile\_recycle}.
  \item
  \texttt{Y\_pc} is calculated by \texttt{neteconomy} and used for
  \texttt{quantile\_recycle}.
\item
  \texttt{country\_pc\_dividend} is calculated by
  \texttt{revenue\_recycle} and is used for \texttt{quantile\_recycle}.
\item
  \texttt{tax\_pc\_revenue} is calculated by \texttt{revenue\_recycle}
  and is used for \texttt{quantile\_recycle}.
\item
  \texttt{qcpc\_post\_recycle} is calculated by \texttt{quantile\_recycle}
  and is used for \texttt{welfare}.
\end{itemize}


\end{document}
